% Please do not change the document class
\documentclass{scrartcl}

% Please do not change these packages
\usepackage[hidelinks]{hyperref}
\usepackage[none]{hyphenat}
\usepackage{setspace}
\doublespace

% You may add additional packages here
\usepackage{amsmath}

% Please include a clear, concise, and descriptive title
\title{CPD Report}

% Please do not change the subtitle
\subtitle{COMP230- Ethics and Professionalism}

% Please put your student number in the author field
\author{1607804}

\begin{document}

\maketitle

%abstract{Please include an abstract of at most 100 words (these do not count towards your word count).}

\section{Key skill: The ability to focus on one piece of work during a short work session.}
When I sit down after university to start doing work I find it difficult to stay focused on one piece of work. I find myself working on multiple different projects in one short session. This results in more time spent on setting up the various projects than I spent actually improving the projects. The ability to stay focused on one aspect of my work at a time will increase the rate I complete any assignemnts I am given. In addition, it may make me less error prone as I don't have my head focused in two different code bases. I have often found myself flicking between doing my COMP220 and COMP230 projects, with the loading times on my laptop being very slow I have wasted a lot of time this semester watching unreal and visual studios load. This has reduced how fast I have been at completeting work this semester. Reducing the overall time I have been able to spend on each project. The smart action I will take is: I will create a plan at the start of each week, structuring what modules I will work on each day. This way when I come to do work during the week, I can refere to my weekly plan and just work on the module I have planned to work on. I will measure my succsess by counting how many times I deviate from my given plan, the fewer the better. I will do this every week for the next semster and see how succsessful it is at improving the quantity and quality of my work. 

\section{Key skill: To better control my emotitions.}
In the games industry, being able to control one's emotions when dealing with colleagues is important to keep an agile workflow. Not being able to control emotions such as frustration may lead to anger. In a diverse workforce like the games industry, it is common to find people who disagree with your ideas or opinions in these cases losing control may lead to a worsening of interpersonal relationships within a team. This will cause communication to worsen and may slow up the workflow for everyone. This semester, some peers within my team have expressed opinions about my work. I found these opinions to be wrong, I let my emotions get the better of me and had a small argument. Afterwards, communication with certain peers worsened for a short period of time. In hindsight, if I had calmly shown my peers the work I had done, disproving their opinions it would have been far more effective and would not have caused a strain on the group's dynamics . To address this issues I will learn three mental calming techniques to practice when I experience adverse emotions. Firstly, I will try the focused breathing, where I take deep slow breaths. Secondly, I will use imagery of a relaxing experience I have had in the past. Thirdly I will use the progressive muscle relaxing, where I flex and relax muscles in my body.  I will count this strategy a success if I avoid feeling intense adverse emotions. I will aim to practice these techniques next term during any periods of particular strain. If they're not effective I will perform more extensive research into other methods.

\section{Key skill: Improve my ability to communicate a summary of a task I have performed.}
Part of an agile workflow are daily standups. In these standups one is meant to communicate to the rest of the team the tasks they have completed since the last standup they have performed. Stand up are often used in the games industry and are meant to be a short concise summary of the tasks they have been working on. If a summary is incomplete other memebers of the team may interepret wrong or that the one speaking has not done any work. This may dissrupt the work flow due to mis comunication. In addition, it could aslo cause memebers to form false opinions about the quantiy of work their peers have done. I believe my inability to summarise the tasks I performed led to my peers forming false opinions about the work I had done for the group game project. This led to the confrontation I disscussed above. My team thought they could not progress on their work because my inability to communicate effectively that I had achieved the work they relied on. By writing a summary of objectives I have achieved at the end of each day I will be able to better recolect the progress I have made. In addition, by writing down the summaries following templates found online I will increase my ability to summarise well. To measure the sucsess of my plan I will ask my peers after a standup and ask them to rate me on how well I performed in the standup. If feedback is negative I will try to adapt using any feedback they give me. I will write my summaries at the end of each day I have done work for the next semester. After this time, I will try and perform standups without writing down summaries each day. If I feel the quality of my communication suffers from not doing them, I will resume the practice. 

\section{Key skill: I will deepen my knowledge of the c\# programming language.}
When searching through jobs available in the games industry, I found lots of adverts for c\# programmers wanted. Lots of high-level APIs use languages such as c\# to streamline the production cycle. Even if I wanted to work on the lower level components of a piece of software, knowledge of the programming language used on the front end may be necessary. An example of this is the unity engine. the core game engine is written in c/c++ with the users scripting in c\# or other languages. During this semester, I tried to help a friend with some basic scripting in c\#. After some struggling, I realised that I need to plug this large gap in my knowledge. While I did understand some parts of it through my knowledge of c++, there is some functionality that is completely different and I found myself unable to help my peer. While this did not affect my work this semester, it did highlight a gap in my knowledge. If this were to happen during a job it could be a much bigger problem. To improve my knowledge of c\# I will complete three episode of a c\# Pluralsight tutorial a week. This will allow me to deepen my knowledge each week without disrupting any other tasks I might have. I will do this for 2 months. After this time I will try writing some basic programs in c\# to measure my success and continue my learning.

\section{}

\bibliographystyle{ieeetran}
\bibliography{references}

\end{document}
