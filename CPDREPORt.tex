% Please do not change the document class
\documentclass{scrartcl}

% Please do not change these packages
\usepackage[hidelinks]{hyperref}
\usepackage[none]{hyphenat}
\usepackage{setspace}
\doublespace

% You may add additional packages here
\usepackage{amsmath}

% Please include a clear, concise, and descriptive title
\title{CPD Report}

% Please do not change the subtitle
\subtitle{COMP230- Ethics and Professionalism}

% Please put your student number in the author field
\author{1607804}

\begin{document}

\maketitle

%abstract{Please include an abstract of at most 100 words (these do not count towards your word count).}

\section{Key skill: The ability to focus on one piece of work during a short work session.}
When I sit down after university to start doing work I find it difficult to stay focused on one piece of work. I find myself working on multiple different projects in one short session. This results in more time spent on setting up the various projects then I spent actually improving the projects. The ability to stay focused on one aspect of my work at a time will increase the rate I complete any assignments I am given. In addition, it may make me less error prone as I don't have my head focused on two different code bases. I have often found myself flicking between doing my COMP220 and COMP230 projects, with the loading times on my laptop, is very slow I have wasted a lot of time this semester watching unreal and visual studios load. This has reduced how fast I have been at completing work this semester. Reducing the overall time I have been able to spend on each project. I will create a plan at the start of each week, structuring what modules I will work on each day. This way when I come to do work during the week, I can refer to my weekly plan and just work on the module I have planned to work on. I will measure my success by counting how many times I deviate from my given plan, the fewer the better. I will do this every week for the next semester and see how successful it is in improving the quantity and quality of my work. 

\section{Key skill: To better control my emotions.}
In the games industry, being able to control one's emotions when dealing with colleagues is important to keep an agile workflow. Not being able to control emotions such as frustration may lead to anger. In a diverse workforce like the games industry, it is common to find people who disagree with your ideas or opinions. Losing control may lead to a worsening of interpersonal relationships within a team. This will cause communication to worsen and may slow up the workflow for everyone. This semester, some peers within my team have expressed opinions about my work. I found these opinions to be wrong, I let my emotions get the better of me and had a brief verbal conflict. Afterwards, communication with certain peers worsened for a short period of time. In hindsight, if I had calmly shown my peers the work I had done, disproving their opinions it would have been far more effective and would not have caused a strain on the group's dynamics. To address this issues I will learn three mental calming techniques to practice when I experience adverse emotions. Firstly, I will try the focused breathing, where I take deep slow breaths. Secondly, I will use imagery of a relaxing experience I have had in the past. Thirdly I will use the progressive muscle relaxing, where I flex and relax muscles in my body.  I will count this strategy a success if I avoid feeling intense adverse emotions. I will aim to practice these techniques next term during any periods of particular strain. If they're not effective I will perform more extensive research into other methods.

\section{Key skill: Improve my ability to communicate a summary of a task I have performed.}
Part of an agile workflow is daily standups. In these standups, one is meant to communicate to the rest of the team the tasks they have completed since the last standup they have performed. Standups are often used in the games industry and are meant to be a short concise summary of the tasks they have been working on. If a summary is incomplete it may lead to misinterpretation or the assumption the one speaking has done too little work. This could cause a disruption to the projects workflow. In addition, it could also cause members to form false opinions about the quantity of work their peers have done. I believe my inability to summarise the tasks I performed led to my peers forming false opinions about the work I had done for the group game project. This led to the confrontation I discussed above. My inability to communicate the tasks I completed held up others as they relied on certain tasks to be completed. At the end of each day, I will write a summary of the work I have done that day. This will help in recalling any progress I have made . In addition, I hope I will increase the quality of my summaries by following good templates I find online. To measure the success of my plan, I will ask my peers for their opinions on my contribution to each standup. If the feedback is negative, I will try to adapt using any feedback they give me. I start next semester and write them at the end of each working day. After this time, I will try and perform standups without writing down summaries. If I feel the quality of my communication suffers from not doing them, I will resume the practice. 

\section{Key skill: I will deepen my knowledge of the c\# programming language.}
When searching for the jobs currently available in the games industry, I found lots of adverts for c\# programmers wanted. Lots of high-level APIs use languages such as c\# to streamline the production cycle. Even if I wanted to work on the lower level components of a piece of software, knowledge of the programming language used on the front end may be necessary. An example of this is the unity engine. The core game engine is written in c/c++ with the users scripting in c\# or other languages. During this semester, I tried to help a friend with some basic scripting in c\#. After some struggling, I realised that I need to plug this large gap in my knowledge. While I did understand some parts of it through my knowledge of c++, there is some functionality that is completely different and I found myself unable to help my peer. While this did not affect my work this semester, it did highlight a gap in my knowledge. If this were to happen during a job it could be a much bigger problem. To improve my knowledge of c\# I will complete three episodes of a c\# Pluralsight tutorial a week. This will allow me to deepen my knowledge each week without disrupting any other tasks I might have. I will do this for two months. After this time I will try writing some basic programs in c\# to measure my success and continue my learning.

\section{Key Skill: Increase my knowledge of strategies for learning }
The software engineering industry is developing at an extremely fast rate. People in this occupation have to adapt and learn the new technologies that come out to remaining competitive in the employment market. By learning different ways to learn, I can identify the most effective techniques for the task. This will make me a more effective learner and will make it easier to stay up to date with the emerging technologies in the field. I have never looked into techniques on how to learn. I have always learnt as I went along, mainly through trial by error. As a result, I think the speed at which I learn has been bottlenecked. Approaching learning in a more structured way may allow me to increase the rate at which I learn. To achieve this I plan on reading one academic paper on learning strategies or similar. This will allow me to grow my knowledge on how to learn and give me enough time between papers to test out techniques. I will measure success by how well I retain the knowledge learned with each technique. I will do this alongside the learning of c\# as I believe they compliment each other both will last two months. 

\bibliographystyle{ieeetran}
\bibliography{references}

\end{document}
